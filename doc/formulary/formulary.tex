
%author: Martina Doku
%date: 2019-10-15

\documentclass[11pt]{article}

\usepackage[utf8]{inputenc}
\usepackage[english]{babel}

\usepackage{amsmath}
\usepackage{hyperref}
\hypersetup{
  colorlinks=true,
  linkcolor=black,
  urlcolor=black
  }
\usepackage{cleveref}

\DeclareMathOperator{\atan2}{atan2}

\begin{document}
\title{Robotics 1 \\ Exercise Solver}
\maketitle
\tableofcontents
\clearpage
\section{Rotation Matrices}\label{sec:rotation}
\subsection{Check if R is a rotation matrix}\label{sec:check_rotation}
To check if R is a rotation matrix we have to check:
\begin{itemize}
  \item det(R) = 1
  \item Orthogonality: $R^TR = I$
  \item Normality: for each column $R_i$ of R, $||R_i|| = 1$
\end{itemize}
\subsection{Rotation direct problem}
To find R from $\theta$ and \textbf{r} we use the Rodrigues' rotation formula:
\begin{equation}
  R(\theta, r)= rr^T+ (I-rr^T)\cos(\theta) + (S(r))\sin(\theta)
\end{equation}
where $S(r)$ is the skew-symmetric matrix of \textbf{r}:
\begin{equation}
  S(r) = \begin{bmatrix}
    0    & -r_3 & r_2  \\
    r_3  & 0    & -r_1 \\
    -r_2 & r_1  & 0
  \end{bmatrix}
\end{equation}
\subsection{Rotation inverse problem}
To find $\theta$ and \textbf{r} from R we fisrt check if there is
a singularity:
\begin{equation}
  \sin(\theta) = \frac{1}{2}\left(\sqrt{(R_{32}-R_{23})^2+(R_{13}-R_{31})^2+(R_{21}-R_{12})^2}\right)
\end{equation}
\subsubsection{Singularity (hence $\sin(\theta) = 0$)}
If it is a singularity we can find \textbf{r} and $\theta$:
if $\theta$ is 0: there is no solution for r.
if $\theta$ is $\pm\pi$:\\
we set $\sin(\theta) = 0$, $\cos(\theta) = -1$ and we find \textbf{r}:
\begin{equation}
  \textbf{r} = \frac{1}{2}\begin{bmatrix}
    \pm \sqrt{R_{11}+1} \\
    \pm \sqrt{R_{22}+1} \\
    \pm \sqrt{R_{33}+1}
  \end{bmatrix}
\end{equation}
To decide the signs of the elements of \textbf{r} we can use the following criteria:
\begin{itemize}
  \item $r_x r_y = R_{12}/2$
  \item $r_x r_z = R_{13}/2$
  \item $r_y r_z = R_{23}/2$
\end{itemize}
\subsubsection{Not singularity}
If the singularity is not present we can find $\theta$ and \textbf{r}:\\
\textbf{Note: we obtain two solutions for $\theta$ and cosequently r}
\begin{align}
  \cos(\theta) & = \left(R_{11}+R_{22}+R_{33}-1\right)                                \\
  \sin(\theta) & = \pm\sqrt{{(R_{32}-R_{23})}^2+(R_{13}-R_{31})^2+(R_{21}-R_{12})^2}  \\
  \theta       & = \atan2\left(\sin\theta,\cos\theta\right) \in \left(-\pi,\pi\right]
\end{align}
\begin{equation}
  \textbf{r} = \frac{1}{2\sin(\theta)}\begin{bmatrix}
    R_{32}-R_{23} \\
    R_{13}-R_{31} \\
    R_{21}-R_{12}
  \end{bmatrix}
\end{equation}

\section{Roll Pitch Yawn}\label{sec:rpy}
\subsection{RPY direct problem}
To find R from $\psi$, $\theta$ and $\phi$ we use the following formula:
\begin{equation}
  R(\psi,\theta,\phi) = R_z(\phi)R_y(\theta)R_x(\psi)
\end{equation}
\textbf{Note: the order of the angles is reversed!}
\subsection{Inverse Problem}
Given a rotation matrix $R$ we can find angles of rotation $\psi$, $\theta$ and $\phi$:
Fisrt we check if there is a singularity (if $R_{32}^2+R_{33}^2 = 0$), we then have
two cases:

\begin{itemize}
  \item \textbf{No Singularity} --- We can find all three parameters of the
        rotational matrix $R$ \begin{align}
          \theta & = \atan2\left(-R_{31},\pm\sqrt{R_{32}^2+R_{33}^2}\right)     \\
          \phi   & = \atan2\left(R_{21}/\cos(\theta),R_{11}/\cos(\theta)\right) \\
          \psi   & = \atan2\left(R_{32}/\cos(\theta),R_{33}/\cos(\theta)\right)
        \end{align}
  \item \textbf{Singularity} --- We cannot find all three angles, only $\theta$ and a combination
        of $\phi$ and $\psi$, the formula for these combinations is:

        \begin{equation}
          \begin{cases}
            \phi - \psi = \atan2 \left\{ R_{2,3}, R_{1,3} \right\} \quad   & \text{if } \theta = \frac{\pi}{2}  \\
            \phi + \psi = \atan2 \left\{  -R_{2,3}, R_{2,2} \right\} \quad & \text{if } \theta = -\frac{\pi}{2}
          \end{cases}
        \end{equation}
\end{itemize}


\section{DH frames}
\subsection{Axis assignment}
\begin{itemize}
  \item \textbf{$z_i$} along the direction of joint i+1.
  \item \textbf{$x_i$} along the common normal between $z_i$ and $z_{i-1}$.
  \item \textbf{$y_i$} completes the right-handed coordinate system.
\end{itemize}
\subsection{DH table}
\begin{itemize}
  \item \textbf{$\theta_i$} angle between $x_{i-1}$ and $x_i$ measured about $z_{i-1}$.
  \item \textbf{$d_i$} distance between $x_{i-1}$ and $x_i$ measured along $z_{i-1}$.
  \item \textbf{$a_i$} distance between $z_{i-1}$ and $z_i$ measured along $x_i$.
  \item \textbf{$\alpha_i$} angle between $z_{i-1}$ and $z_i$ measured about $x_i$.
\end{itemize}
\subsection{Transformation matrix from DH parameters}
\begin{equation}
  ^{i-1}A_i = \begin{bmatrix}
    \cos(\theta_i) & -\sin(\theta_i)\cos(\alpha_i) & \sin(\theta_i)\sin(\alpha_i)  & a_i\cos(\theta_i) \\
    \sin(\theta_i) & \cos(\theta_i)\cos(\alpha_i)  & -\cos(\theta_i)\sin(\alpha_i) & a_i\sin(\theta_i) \\
    0              & \sin(\alpha_i)                & \cos(\alpha_i)                & d_i               \\
    0              & 0                             & 0                             & 1
  \end{bmatrix}
\end{equation}
\subsection{DH parameters from transformation matrix}
First we have to check that the first three by three submatrix is
a rotation matrix
\hyperref[sec:check_rotation]{(see \cref{sec:check_rotation})}.\\
Then we can find the parameters:
\begin{align}
  \theta_i & = \atan2\left(R_{12},R_{11}\right)          \\
  \alpha_i & = \atan2\left(R_{32},R_{33}\right)          \\
  d_i      & = R_{34}                                    \\
  a_i      & = R_{14}\cos(\theta_i)+R_{24}\sin(\theta_i)
\end{align}
\section{Workspace}
\subsection{2-DOF robot}
The primary workspace is defined by two concentric circles of radius $r_1$ and $r_2$ where:
\begin{equation}
  r_1 = |l_1 - l_2|
\end{equation}
\begin{equation}
  r_2 = l_1 + l_2
\end{equation}
\subsection{3-DOF robot}
The primary workspace is defined by two concentric spheres of radius $r_{in}$ and $r_{out}$ where:
\begin{equation}
  r_{in}= l_{min}+l_{med}+l_{max}
\end{equation}
\begin{equation}
  r_{out} = \max(0,l_{max}- l_{med} -l_{min})
\end{equation}
where:
\begin{itemize}
  \item $l_{min}$ is the length of the shortest link
  \item $l_{med}$ is the length of the medium link
  \item $l_{max}$ is the length of the longest link
\end{itemize}
\end{document}
